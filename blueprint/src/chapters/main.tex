\chapter{Main Theorems}

\begin{theorem}
\label{thm:big_computational_step}
There exists a solution tree that spans all of
\begin{align*}
\theta_1,\theta_2&\in[0,2\pi/15] \subset [0,0.42], \\
\varphi_1&\in [0,\pi] \subset [0,3.15],\\
\varphi_2&\in [0,\pi/2] \subset [0,1.58],\\
\alpha &\in [-\pi/2,\pi/2] \subset [-1.58,1.58].
\end{align*}
such that every point is covered by a leaf equipped with enough
evidence to allow the rational global theorem or rational local theorem to apply.
\end{theorem}

\begin{theorem}
\label{thm:no_nopert_tight_view_pose}
\lean{no_nopert_tight_view_pose}
\uses{def:noperthedron}
\leanok
There does not in fact exist a noperthedron Rupert solution with
\begin{align*}
\theta_1,\theta_2&\in[0,2\pi/15] \subset [0,0.42], \\
\varphi_1&\in [0,\pi] \subset [0,3.15],\\
\varphi_2&\in [0,\pi/2] \subset [0,1.58],\\
\alpha &\in [-\pi/2,\pi/2] \subset [-1.58,1.58].
\end{align*}
\end{theorem}

\begin{proof}
\uses{thm:big_computational_step, thm:global_rational, thm:local_rational}
By \ref{thm:big_computational_step}, every point in the above 5-dimensional
interval is covered by the rational global theorem (Theorem~\ref{thm:global_rational})
or the rational local theorem (Theorem~\ref{thm:local_rational}).
\end{proof}

\begin{theorem}
\label{thm:no_nopert_view_pose}
\lean{no_nopert_view_pose}
\leanok
There is no view pose that makes the noperthedron have the Rupert property.
\end{theorem}
\begin{proof}
\uses{thm:no_nopert_tight_view_pose, cor:rupert_tightening}
\leanok
Theorem~\ref{thm:no_nopert_tight_view_pose} says there is no tight view pose
that makes the noperthedron Rupert. Corollary~\ref{cor:rupert_tightening}
says that this suffices for the general case.
\end{proof}


\begin{theorem}
\label{thm:no_nopert_rot_pose}
\lean{no_nopert_rot_pose}
\leanok
There is no purely rotational pose that makes the noperthedron have the Rupert property.
\end{theorem}
\begin{proof}
\uses{thm:view_pose_of_pose, thm:no_nopert_view_pose}
\leanok
Suppose there were a purely rotational pose. Then convert that to an equivalent view pose
with Theorem~\ref{thm:view_pose_of_pose} and
appeal to Theorem~\ref{thm:no_nopert_view_pose}.
\end{proof}


\begin{theorem}
\label{thm:no_nopert_pose}
\lean{no_nopert_pose}
\leanok
There is no pose that makes the noperthedron have the Rupert property.
\end{theorem}
\begin{proof}
\uses{lemma:nopert_point_symmetric, thm:no_nopert_rot_pose, thm:rupert_implies_rot_rupert}
\leanok
By Theorem~\ref{thm:rupert_implies_rot_rupert}, we need only show that the noperthedron
is pointsymmetric to see that if it is Rupert, then it must be Rupert via a purely rotational pose.
But Lemma~\ref{lemma:nopert_point_symmetric} shows exactly this. And yet we know via
Theorem~\ref{thm:no_nopert_rot_pose} that the noperthedron is not rotationally Rupert, so we have
a contradiction, hence the noperthedron has no pose that makes it Rupert.
\end{proof}


\begin{theorem}
\uses{def:noperthedron}
\label{thm:nopert_not_rupert_set}
\lean{nopert_not_rupert_set}
\leanok
The noperthedron is not a Rupert set.
\end{theorem}
\begin{proof}
\uses{thm:no_nopert_pose}
\leanok
By Theorem~\ref{thm:no_nopert_pose}, there is no pose that makes the noperthedron a Rupert set.
\end{proof}

\begin{theorem}
\label{thm:nopert_not_rupert}
\lean{nopert_not_rupert}
\leanok
The noperthedron is not a Rupert polyhedron.
\end{theorem}

\begin{proof}
\leanok
\uses{thm:rupert_iff_rupert_set, thm:nopert_not_rupert_set}
By Theorem~\ref{thm:rupert_iff_rupert_set} it suffices to show that the convex hull of
the noperthedron vertices is not a Rupert set. But this is exactly what Theorem~\ref{thm:nopert_not_rupert_set} shows.
\end{proof}
