\chapter{The Noperthedron}

\section{Definition of the Noperthedron}

We define three points $C_1,C_2,C_3\in \mathbb{Q}^3$.
\[
    C_1\coloneqq
        \frac{1}{259375205}
        \begin{pmatrix}
        {152024884} \\ 0 \\ {210152163}
        \end{pmatrix},
\qquad
    C_2\coloneqq \frac{1}{10^{10}}
        \begin{pmatrix}
        6632738028 \\ 6106948881 \\ 3980949609
        \end{pmatrix},
\]
\[
    C_3\coloneqq
        \frac{1}{10^{10}}
        \begin{pmatrix}
        8193990033 \\ 5298215096 \\ 1230614493
        \end{pmatrix}.
\]

\begin{lemma}
\label{c1_c2_c3_norms}
\lean{Nopert.c1_norm_one, Nopert.c2_norm_bound, Nopert.c3_norm_bound, Nopert.C15}
\leanok
$\| C_1 \| = 1$,
${98 \over 100} < \| C_2 \| < {99 \over 100}$, and
${98 \over 100} < \| C_3 \| < {99 \over 100}$.
\end{lemma}
\begin{proof}
\leanok
Trivial arithmetic.
\end{proof}

\begin{lemma}
\label{lem:radius_noperthedron_one}
\lean{Nopert.noperthedron_radius_one}
\leanok
The radius of the Noperthedron is one.
\end{lemma}
\begin{proof}
\leanok
\uses{c1_c2_c3_norms, thm:pointsymmetrize_pres_radius, thm:polyhedron_radius_def, lemma:half_nopert_verts_norm_le_one}
By \cref{c1_c2_c3_norms}, \cref{thm:pointsymmetrize_pres_radius}, \cref{thm:polyhedron_radius_def},
and \cref{lemma:half_nopert_verts_norm_le_one}.
\end{proof}

Rotations about the $x, y, z$ axes $R_x,R_y,R_z:$  $\mathbb{R}\to \mathbb{R}^{3\times 3}$
are defined in the usual way:
    \[
            R_x(\alpha)\coloneqq
        \begin{pmatrix}
            1 & 0 & 0\\
            0 & \cos\alpha & -\sin\alpha\\
            0 & \sin\alpha & \cos\alpha
        \end{pmatrix},
        \hspace{1cm}
        R_y(\alpha)\coloneqq
        \begin{pmatrix}
            \cos\alpha & 0 & -\sin\alpha\\
            0 & 1 & 0\\
            \sin\alpha & 0 & \cos\alpha
        \end{pmatrix},
    \]
    \[
        R_z(\alpha)\coloneqq
        \begin{pmatrix}
            \cos\alpha & -\sin\alpha &0\\
            \sin\alpha & \cos\alpha &0\\
            0 & 0 & 1
        \end{pmatrix}.
    \]

Where Steininger and Yurkevich define a 30-element set $C_30$
\[
    \mathcal{C}_{30} \coloneqq \left\{(-1)^\ell R_z\left(\frac{2\pi k}{15}\right) \colon k=0,\dots,14; \ell=0,1\right\}.
\]
of rotations, we instead define
\begin{definition}
  \label{def:C15}
  \leanok
  \lean{Nopert.C15}
\[
    \mathcal{C}_{15} \coloneqq \left\{ R_z\left(\frac{2\pi k}{15}\right) \colon k=0,\dots,14 \right\}.
\]
\end{definition}
without point-symmetricness `baked in' as it is in $C_{30}$. It's more convenient for the formalization to apply $C_{15}$ to the points $C_1, C_2, C_3$, and then point-symmetrize that set afterwards.

\begin{definition}
\label{def:pointsymmetric}
\lean{PointSym}
\leanok
A set $S \subseteq \R^3$ is {\em point-symmetric} if $x \in S$ implies $-x \in S$.
\end{definition}

\begin{definition}
\label{def:pointsymmetrize}
\lean{pointsymmetrize}
\leanok
The {\em pointsymmetrization} of a collection of vertices $v_1, \ldots, v_n \in \R^3$
is $v_1, \ldots, v_n, -v_1, \ldots, -v_n$.
\end{definition}

We write $\mathcal{C}_{15} \cdot P = \{c P \,\text{ for } \, c \in \mathcal{C}_{15}\}$ for the orbit of $P$ under the action of $\mathcal{C}_{15}$.

\begin{definition}
\label{def:noperthedron}
\lean{halfNopertVerts, nopertVerts, nopert}
\uses{def:pointsymmetrize,def:C15}
\leanok
The Noperthedron is polyhedron given by the vertex set that is the
pointsymmetrization of
\[\mathcal{C}_{15} \cdot C_1 \cup \mathcal{C}_{15} \cdot C_2 \cup \mathcal{C}_{15} \cdot C_3\]
\end{definition}

\begin{lemma}
\label{lemma:half_nopert_verts_norm_le_one}
\lean{half_nopert_verts_norm_le_one}
\leanok
The norm of any vertex in the prepointsymmetrized version of the Noperthedron is no more than 1.
\end{lemma}
\begin{proof}
\leanok
Evident from definitions.
\end{proof}

\begin{lemma}
\label{lemma:pointsymmetrization_is_pointsym}
\lean{pointsymmetrize_is_pointsym}
\leanok
The pointsymmetrization of any set is point-symmetric.
\end{lemma}
\begin{proof}
\leanok
Evident from definitions.
\end{proof}

\begin{lemma}
\label{lemma:nopert_point_symmetric}
\lean{nopert_point_symmetric}
\leanok
\uses{def:pointsymmetric, def:noperthedron}
The noperthedron is point-symmetric.
\end{lemma}
\begin{proof}
\leanok
\uses{lemma:pointsymmetrization_is_pointsym}
Follows from Lemma~\ref{lemma:pointsymmetrization_is_pointsym}.
\end{proof}

\section{Refined Rupert's property for the Noperthedron}

\begin{lemma} \label{lem:symmetries}
\leanok
\lean{Tightening.lemma7_1,Tightening.lemma7_2,Tightening.lemma7_3}
Let $\PPP = \NOP$, then for all $\theta, \varphi, \alpha \in \R$, the following three identities hold (as sets):
\begin{align*}
    M({\theta+2\pi/15,\varphi})\cdot \PPP &=M(\theta, \phi) \cdot \PPP,\\
    R(\alpha+\pi)M(\theta, \phi) \cdot \PPP &=R(\alpha)M(\theta, \phi) \cdot \PPP,\\
    \begin{pmatrix}
        1&0\\
        0&-1
    \end{pmatrix}
    M(\theta, \phi) \cdot \PPP&=
    M({\theta+\pi/15,\pi-\varphi}) \cdot \PPP.
\end{align*}
\end{lemma}
\begin{proof}
See \cite{polyhedron.without.rupert}, Lemma 7.
\end{proof}


\begin{corollary}
\label{cor:rupert_tightening}
\lean{Tightening.rupert_tightening}
\leanok
If the noperthedron is Rupert, then there exists a solution with
\begin{align*}
\theta_1,\theta_2&\in[0,2\pi/15] \subset [0,0.42], \\
\varphi_1&\in [0,\pi] \subset [0,3.15],\\
\varphi_2&\in [0,\pi/2] \subset [0,1.58],\\
\alpha &\in [-\pi/2,\pi/2] \subset [-1.58,1.58].
\end{align*}
\end{corollary}

\begin{proof}
\uses{lem:symmetries}
See \cite{polyhedron.without.rupert}, Lemma 8.
\end{proof}
