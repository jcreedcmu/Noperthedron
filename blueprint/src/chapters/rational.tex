\chapter{Rational Versions}

\begin{definition} \label{dfn:sin_cos_approx}
\leanok
\lean{RationalApprox.sinℚ,RationalApprox.cosℚ}
    We define the two functions $\ssin, \scos: \R \to \R$ by:
    \begin{align*}
    \ssin(x) & \coloneqq x-\frac{x^3}{3}+\frac{x^5}{5!}\mp \dots +\frac{x^{25}}{25!},\\
    \scos(x) & \coloneqq 1-\frac{x^2}{2}+\frac{x^4}{4!}\mp\dots +\frac{x^{24}}{24!}.
    \end{align*}
    Further, by replacing $\sin,\cos$ with $\ssin,\scos$ we define the functions
\[
    R_{\Q}(\alpha), R'_{\Q}(\alpha), X_{\Q}(\theta, \phi), M_{\Q}(\theta, \phi), M_{\Q}^{\theta}(\theta,\varphi),M_{\Q}^{\phi}(\theta,\varphi).
\]
\end{definition}

\begin{lemma} \label{lem:sin27cos26}
\leanok
\lean{RationalApprox.sinℚ_approx,RationalApprox.cosℚ_approx}
    \[
        |\ssin(x)-\sin(x)|\leq \frac{|x|^{27}}{27!} \quad \text{and} \quad
    |\scos(x)-\cos(x)|\leq \frac{|x|^{26}}{26!}.
    \]
\uses{dfn:sin_cos_approx}
\end{lemma}
\begin{proof}
Appeal to Taylor series bounds, using the fact that all absolute values of
higher derivatives of sine and cosine never exceed 1.
\end{proof}

\begin{lemma} \label{lem:kappa7}
\leanok
\lean{RationalApprox.sinℚ_approx',RationalApprox.cosℚ_approx'}
    For every $x\in [-4,4]$ it holds that
    \[
    |\ssin(x)-\sin(x)| \leq \frac{\kappa}{7} \quad \text{and} \quad |\scos(x)-\cos(x)|\leq \frac{\kappa}{7}.
    \]
\end{lemma}
\begin{proof}
\uses{lem:sin27cos26}
\leanok
Straightforward numerical calculation from Lemma~\ref{lem:sin27cos26}.
\end{proof}

\begin{lemma} \label{lem:A_le_deltamn}
  \lean{RationalApprox.norm_le_delta_sqrt_dims}
  \leanok
Let $A = (a_{i,j})_{1 \leq i \leq m,\ 1 \leq j \leq n} \in \R^{m \times n}$ and $\delta >0$. Assume that $|a_{i,j}| \leq \delta$. Then it holds that $\|A\| \leq \delta \sqrt{mn}.$
\end{lemma}

\begin{proof}
    For any $v\in \R^n$ we have
    \begin{align*}
        \|Av\|^2 &=\sum_{i=1}^m \left(\sum_{j=1}^na_{i,j}v_j\right)^2 \leq \sum_{i=1}^m\left(\sum_{j=1}^n \delta |v_j|\right)^2 = \delta^2 m\left(\sum_{j=1}^n |v_j|\right)^2 \leq \delta^2 m n \|v\|^2
    \end{align*}
    using the Cauchy-Schwarz inequality. Dividing by $\|v\|$ and taking the square root proves the claim.
\end{proof}

\begin{lemma}\label{lem:dist_lt_kappa}
\leanok
\lean{RationalApprox.norm_matrix_actual_approx_le_kappa}
    Let $A(x,y)$ be an $m\times n$ matrix with $1 \leq m,n\leq 3$ such that every entry is in $[-1,1]$.of the form $a_1(x)\cdot a_2(y)$ where $a_i(z) \in [-1,1]$.
    Define $A_{\mathbb{Q}}(x,y)$ by replacing $\sin$ with $\ssin$ and $\cos$ with $\scos$. Then for every $x,y\in[-4,4]$ it holds that
    $\|A(x,y)-A_{\mathbb{Q}}(x,y)\|\leq\kappa$.
\end{lemma}

\begin{proof}
\uses{lem:kappa7,lem:A_le_deltamn}
    We've replaced the assumption $a_i(z)\in \{0,1,-1,\pm\sin(z),\pm\cos(z)\}$  in See \cite{polyhedron.without.rupert}'s
    Lemma~40 with $a_i(z)\in [-1,1]$.

    By assumption, for fixed $x,y$ every entry of $A(x,y)-A_{\mathbb{Q}}(x,y)$ is of the form
    $a b - \widetilde{a}\widetilde{b}$ for some $a,b\in[-1,1]$ and $|a-\widetilde{a}|,|b-\widetilde{b}|\leq \kappa/7$ by \cref{lem:kappa7}. This implies that
    \begin{align*}
      |ab-\widetilde{a}\widetilde{b}|&\leq |a b-a\widetilde{b}|+|a \widetilde{b}-\widetilde{a}\widetilde{b}|
      =|a|\cdot |b-\widetilde{b}|+|\widetilde{b}|\cdot |a-\widetilde{a}| \leq 1\cdot \kappa/7+(1+\kappa/7) \cdot \kappa/7 <\kappa/3.
    \end{align*}
    So we can apply \cref{lem:A_le_deltamn} and obtain that $\|A(x,y)-A_{\Q}(x,y)\|<\kappa/3\cdot \sqrt{3\cdot 3}=\kappa$.
\end{proof}


\begin{corollary} \label{corr:kappa1kappa}
    Let $\alpha,\theta,\phi\in [-4,4]$. Then it holds that
    \begin{align*}
        \|R(\alpha)-R_{\Q}(\alpha)\|, \|R'(\alpha)-R_{\Q}'(\alpha)\|,\|X(\theta,\phi)-X_{\Q}(\theta, \phi)\|, \|M(\theta, \phi)-M_{\Q}(\theta, \phi)\|, \\
        \|M^\theta(\theta,\phi)-M_{\Q}^\theta(\theta,\phi)\|,
        \|M^\phi(\theta,\phi) - M_{\Q}^\phi(\theta,\phi)\|\leq \kappa.
    \end{align*}
    Moreover,
    \[
        \|R_{\Q}(\alpha)\|, \|R'_{\Q}(\alpha)\|, \|X_{\Q}(\theta, \phi)\|, \|M_{\Q}(\theta, \phi)\|, \|M_{\Q}^{\theta}(\theta,\varphi)\|, \|M_{\Q}^{\phi}(\theta,\varphi)\| \leq 1+\kappa
    \]
\end{corollary}

\begin{proof}
\uses{lem:dist_lt_kappa,lem:RaRalpha}
    The first statement is a direct application of \cref{lem:dist_lt_kappa} and the second statement follows immediately after using \cref{lem:RaRalpha} and the triangle inequality.
\end{proof}

\begin{lemma} \label{lem:A1AnB1Bn}
    For $1 \leq i \leq n$ let $(A_i,B_i)$ be pairs of real matrices, such that for each $i$ the dimensions of $A_i$ and $B_i$ are equal. Assume moreover that the products $A_1\cdots A_n$ and $B_1 \cdots B_n$ are well defined. Finally, assume that $\|A_i-B_i\|\leq \kappa$ and let $\delta_i\geq \max(\|A_i\|,\|B_i\|,1)$. Then it holds that
    $\|A_1\cdots A_n-B_1\cdots B_n\|\leq n\kappa\cdot \delta_1\cdots \delta_n$.
\end{lemma}

\begin{proof}
See \cite{polyhedron.without.rupert}, Lemma 42.
\end{proof}


\begin{lemma} \label{lem:boundskappa}
    Let $\alpha, \theta, \phi \in [-4,4]$, $P\in \R^3$ with $\|P\| \leq 1$ and let $\widetilde{P}$ be a $\kappa$-rational approximation of $P$. Set $M = M(\theta, \phi)$ and $M_{\Q} = M_{\Q}(\theta, \phi)$, $M^\theta = M^\theta(\theta, \phi)$, $M^\theta_{\Q} = M^\theta_{\Q}(\theta, \phi)$, $M^\phi = M^\phi(\theta, \phi)$, $M^\phi_{\Q} = M^\phi_{\Q}(\theta, \phi)$ as well as $R = R(\alpha)$, $R_{\Q} = R_{\Q}(\alpha)$, $R' = R'(\alpha)$, $R'_{\Q} = R'_{\Q}(\alpha)$. Finally let $w \in \R^2$ with $\|w\| = 1$. Then:
    \begin{align}
        | \langle M P, w\rangle - \langle M_{\Q} \widetilde{P}, w\rangle | & \leq 3\kappa, \label{eq:boundskappa1} \\
        | \langle M^\theta P, w\rangle - \langle M^\theta_{\Q} \widetilde{P}, w\rangle | & \leq 3\kappa,\\
        | \langle M^\phi P, w\rangle - \langle M^\phi_{\Q} \widetilde{P}, w\rangle | & \leq 3\kappa,\\
        | \langle R M P, w\rangle - \langle R_{\Q} M_{\Q} \widetilde{P}, w\rangle | & \leq 4\kappa,\label{eq:boundskappa4} \\
        | \langle R' M P, w\rangle - \langle R'_{\Q} M_{\Q} \widetilde{P}, w\rangle | & \leq 4\kappa,\\
        | \langle R M^\theta P, w\rangle - \langle R_{\Q} M^\theta_{\Q} \widetilde{P}, w\rangle | & \leq 4\kappa,\\
        | \langle R M^\phi P, w\rangle - \langle R_{\Q} M^\phi_{\Q} \widetilde{P}, w\rangle | & \leq 4\kappa.
    \end{align}
\end{lemma}
\begin{proof}
\uses{lem:A1AnB1Bn,corr:kappa1kappa}
See \cite{polyhedron.without.rupert}, Lemma 44.
\end{proof}


\begin{theorem}[Rational Global Theorem] \label{thm:global_rational}
    Let $\PPP$ be a pointsymmetric convex polyhedron with radius $\rho =1$ and $\widetilde{\PPP}$ a $\kappa$-rational approximation. Let $\widetilde{S} \in \widetilde{\PPP}$. Further let $\epsilon>0$ and $\thetab_1,\phib_1,\thetab_2,\phib_2,\alphab \in \Q \cap [-4,4]$.
    Let $w\in\Q^2$ be a unit vector. Denote $\Mib \coloneqq M_{\Q}(\thetab_1, \phib_1)$, $ \Miib \coloneqq M_{\Q}(\thetab_2, \phib_2)$ as well as $\Mib^{\theta} \coloneqq M_{\Q}^\theta(\thetab_1, \phib_1)$, $\Mib^{\phi} \coloneqq M_{\Q}^\phi(\thetab_1, \phib_1)$ and analogously for $\Miib^{\theta}, \Miib^{\phi}$. Finally set
    \begin{align*}
        G^{\Q}& \coloneqq \langle R_{\Q}(\alphab) \Mib \widetilde{S},w \rangle - \epsilon\cdot\big(|\langle R_{\Q}'(\alphab)  \Mib \widetilde{S},w \rangle|+|\langle R_{\Q}(\alphab) \Mib^\theta \widetilde{S},w \rangle|+|\langle R_{\Q}(\alphab) \Mib^\phi \widetilde{S},w \rangle|\big) \\
        & \hspace{11cm}- 9\epsilon^2/2 - 4\kappa ( 1 + 3 \epsilon),\\
        H^{\Q}_P & \coloneqq \langle \Miib P,w \rangle + \epsilon\cdot\big(|\langle \Miib^\theta P,w \rangle|+|\langle  \Miib^\varphi P,w \rangle|\big) + 2\epsilon^2 + 3\kappa( 1+2\epsilon).
    \end{align*}
    If $G^{\Q}>\max_{P\in \widetilde{\PPP}} H^{\Q}_P$ then there does not exist a solution to Rupert's condition to $\PPP$ with
    \[
    (\theta_1,\varphi_1,\theta_2,\varphi_2,\alpha) \in [\thetab_1\pm\epsilon,\phib_1\pm\epsilon,\thetab_2\pm\epsilon,\phib_2\pm\epsilon,\alphab\pm\epsilon].
    \]
\end{theorem}

\begin{proof}
\uses{thm:global,lem:boundskappa}
\end{proof}

\begin{lemma} \label{lem:ekspanningespanning}
    Let $P_1, P_2, P_3 \in \R^3$ with $\|P_i\| \leq 1$ and $\widetilde{P}_1, \widetilde{P}_2, \widetilde{P}_3 \in \Q^3$ be their $\kappa$-rational approximations. Assume that $\widetilde{P}_1, \widetilde{P}_2, \widetilde{P}_3$ are $\epsilon$-$\kappa$-spanning for some $\theta, \phi \in \Q \cap [-4,4]$, then $P_1, P_2, P_3$ are $\epsilon$-spanning for $\theta, \phi$.
\end{lemma}
\begin{proof}
See \cite{polyhedron.without.rupert}, Lemma 46.
\uses{lem:A1AnB1Bn,lem:eps-spanning}
\end{proof}


\begin{lemma} \label{lem:boundskappa3}
    Let $P,Q \in \R^3$ with $\|P\|,\|Q\|\leq 1$ and $\widetilde{P},\widetilde{Q}$ some respective $\kappa$-rational approximations. Moreover, let $\alpha, \theta, \phi \in \R \in [-4,4]$ and set $X = X(\theta, \phi)$, $X_{\Q} = X_{\Q}(\theta, \phi)$ as well as $M = M(\theta, \phi)$, $M_{\Q} = M_{\Q}(\theta, \phi)$. Then
    \begin{align}
        |\langle X, P \rangle - \langle X_{\Q}, \widetilde{P} \rangle| & \leq 3 \kappa, \label{eq:boundskappa3.1}\\
        |\langle MP, MQ \rangle - \langle M_{\Q} \widetilde{P}, M_{\Q}\widetilde{Q} \rangle| & \leq 5 \kappa, \label{eq:boundskappa3.3}\\
        |\| M Q \| - \| M_{\Q}\widetilde{Q} \| | & \leq 3 \kappa.\label{eq:boundskappa3.2}
    \end{align}
\end{lemma}
\begin{proof}
See \cite{polyhedron.without.rupert}, Lemma 49.
\uses{lem:A1AnB1Bn}
\end{proof}


\begin{corollary} \label{corr:deltakappa}
    In the setting of  \cref{lem:boundskappa3} let additionally $\thetab, \phib \in \R \cap [-4,4]$ and set $\overline{M} = M(\thetab, \phib)$, $\overline{M}_{\Q} = M_{\Q}(\thetab, \phib)$. Then
    \[
    |\| R(\alpha) M P - \overline{M} Q\|- \| R_{\Q}(\alpha) M_{\Q} \widetilde{P} - \overline{M}_{\Q} \widetilde{Q}\| | \leq 6 \kappa.\label{eq:boundskappa3.4}
    \]
\end{corollary}
\begin{proof}
See \cite{polyhedron.without.rupert}, Corollary 50.
\uses{lem:boundskappa3}
\end{proof}


\begin{corollary} \label{lem:boundskappa4}
    In the setting of \cref{lem:boundskappa3}, let $\sqrt[+]{x}$ be an upper-$\Q$-square-root function and set $\|x\|_{+} \coloneqq \sqrt[+]{\|x\|^2}$. Set
    \[
        A =  \frac{\langle M P, M(P-Q)\rangle - 2 \epsilon  \|P-Q\| \cdot  (\sqrt{2}+\varepsilon)}{ \big(\| M P\|+\sqrt{2} \varepsilon \big) \cdot \big(\|M(P-Q)\|+2 \sqrt{2} \varepsilon\big)}
    \]
    as well as
    \[
        A_{\Q} =         \frac{\langle M_{\Q} \widetilde{P}, M_{\Q} (\widetilde{P}-\widetilde{Q})\rangle - 10\kappa - 2 \epsilon ( \|\widetilde{P}-\widetilde{Q}\|_{+} + 2 \kappa ) \cdot  (\sqrt{2}+\varepsilon)}{ \big(\| M_{\Q} \widetilde{P}\|_{+}+\sqrt{2} \varepsilon + 3\kappa \big) \cdot \big(\|M_{\Q}(\widetilde{P}-\widetilde{Q})\|_{+}+2 \sqrt{2} \varepsilon + 6\kappa\big)}.
    \]
    Then it holds that $A \geq A_{\Q}$.
\end{corollary}
\begin{proof}
See \cite{polyhedron.without.rupert}, Corollary 51.
\uses{lem:boundskappa3}
\end{proof}

\begin{theorem}[Rational Local Theorem] \label{thm:local_rational}
    Let $\PPP$ be a polyhedron with radius $\rho=1$ and $\widetilde{P}_i$ be a $\kappa$-rational approximation of $P_i \in \PPP$. Set $\widetilde{\PPP} = \{\widetilde{P}_i \text{ for } P_i \in \PPP \}$. Let $P_1, P_2, P_3, Q_1, Q_2, Q_3 \in \PPP$ be not necessarily distinct and assume that $P_1, P_2, P_3$ and $Q_1, Q_2, Q_3$ are congruent.
    Let $\epsilon>0$ and $\thetab_1,\phib_1,\thetab_2,\phib_2,\alphab \in \Q \cap [-4,4]$.
    Set $\Xib \coloneqq X_{\Q}(\thetab_1,\phib_1), \Xiib \coloneqq X_{\Q}(\thetab_2,\phib_2)$ as well as $\Mib \coloneqq M_{\Q}(\thetab_1,\phib_1), \Miib \coloneqq M_{\Q}(\thetab_2,\phib_2)$.
    Assume that there exist $\sigma_P, \sigma_Q \in \{0,1\}$ such that
    \[
        (-1)^{\sigma_P} \langle \Xib,\widetilde{P}_i\rangle>\sqrt{2}\varepsilon + 3\kappa \quad \text{and} \quad
        (-1)^{\sigma_Q} \langle \Xiib , \widetilde{Q}_i\rangle>\sqrt{2}\varepsilon + 3\kappa, \tag{A$^{\Q}_\epsilon$}
    \]
    for all $i=1,2,3$.
    Moreover, assume that $\widetilde{P}_1,\widetilde{P}_2,\widetilde{P}_3$ are $\epsilon$-$\kappa$-spanning for $(\thetab_1,\phib_1)$ and that $\widetilde{Q}_1,\widetilde{Q}_2,\widetilde{Q}_3$ are $\epsilon$-$\kappa$-spanning for $(\thetab_2,\phib_2)$. Let $\sqrt[+]{x}$ and $\sqrt[-]{x}$ be upper- and lower-$\Q$-square-root functions, then set $\|Z\|_{+} \coloneqq \sqrt[+]{\|Z\|^2}$ and $\|Z\|_{-} \coloneqq \sqrt[-]{\|Z\|^2}$ for $Z \in \Q^n$.
    Finally, assume that for all $i = 1,2,3$ and any $\widetilde{Q}_j \in \widetilde{\PPP} \setminus \widetilde{Q}_i$ it holds that
    \[
        \frac{\langle \Miib \widetilde{Q}_i,\Miib (\widetilde{Q}_i-\widetilde{Q}_j)\rangle - 10\kappa - 2 \epsilon ( \|\widetilde{Q}_i-\widetilde{Q}_j\|_{+} + 2 \kappa ) \cdot  (\sqrt{2}+\varepsilon)}{ \big(\|\Miib \widetilde{Q}_i\|_{+}+\sqrt{2} \varepsilon + 3\kappa \big) \cdot \big(\|\Miib(\widetilde{Q}_i-\widetilde{Q}_j)\|_{+}+2 \sqrt{2} \varepsilon + 6\kappa\big)} > \frac{\sqrt{5} \epsilon + \delta}{r}, \tag{B$^{\Q}_\epsilon$}
    \]
    for some $r >0$ such that $\min_{i=1,2,3}\| \Miib \widetilde{Q}_i \|_{-} > r + \sqrt{2} \epsilon + 3\kappa$ and for some $\delta \in \R$ with
    \[
        \delta = \max_{i=1,2,3}\left\|R_{\Q}(\alphab) \Mib \widetilde{P}_i-\Miib \widetilde{Q}_i\right\|_{+}/2 + 3\kappa.
    \]
    Then there exists no solution to Rupert's problem $R(\alpha) M(\theta_1,\phi_1)\PPP \subset  M(\theta_2,\phi_2)\PPP^\circ$ with
    \[
        (\theta_1, \phi_1, \theta_2, \phi_2, \alpha) \in [\thetab_1\pm\epsilon,\phib_1\pm\epsilon,\thetab_2\pm\epsilon,\phib_2\pm\epsilon,\alphab\pm\epsilon] \subseteq \R^5.
    \]
\end{theorem}

\begin{proof}
\uses{thm:local,lem:boundskappa3,lem:boundskappa4,corr:deltakappa,lem:ekspanningespanning},
\end{proof}
