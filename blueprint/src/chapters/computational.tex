\chapter{Computational Step}

\begin{theorem}
\lean{exists_solution_table}
\leanok
\label{thm:exists_solution_table}
There exists a valid solution table with some row that covers
\begin{align*}
\theta_1,\theta_2&\in[0,2\pi/15] \subset [0,0.42], \\
\varphi_1&\in [0,\pi] \subset [0,3.15],\\
\varphi_2&\in [0,\pi/2] \subset [0,1.58],\\
\alpha &\in [-\pi/2,\pi/2] \subset [-1.58,1.58].
\end{align*}
\end{theorem}

\begin{proof}
By exhibiting the table and running the validity checking algorithm.
\end{proof}


\begin{theorem}
\leanok
\lean{Solution.Row.valid_imp_not_rupert}
\label{thm:row_valid_imp_not_rupert}
\uses{def:noperthedron}
For any valid row in a valid solution table, there can be no Rupert solution
in the pose interval of that row.
\end{theorem}

\begin{proof}
\uses{thm:global_rational, thm:local_rational, lem:radius_noperthedron_one}
Either appeal recursively to this same theorem if the row splits into other nodes in
the tree, or appeal to the rational global theorem
(Theorem~\ref{thm:global_rational}) or the rational local theorem
(Theorem~\ref{thm:local_rational}) at the leaves.
\end{proof}
