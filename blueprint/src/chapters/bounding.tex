\chapter{Bounding Rotations}

\begin{lemma} \label{lem:RaRalpha}
For any $\alpha, \theta,\varphi \in \R$ and $a \in \{x,y,z\}$ one has $\| R(\alpha)\| = \| R_a(\alpha)\| =\| M(\theta, \phi)\| = 1$.
\end{lemma}
\begin{proof}
See \cite{polyhedron.without.rupert}, Lemma 9.
\end{proof}

\begin{lemma} \label{lem:RaRa}
  \lean{Bounding.RaRaXXX}
  \leanok
Let $\epsilon>0$, $|\alpha-\alphab|\leq\varepsilon$ and $a \in \{x,y,z\}$ then
$\|R_a(\alpha)-R_a({\alphab})\|=\|R(\alpha)-R(\alphab)\| < \varepsilon$.
\end{lemma}
\begin{proof}
See \cite{polyhedron.without.rupert}, Lemma 10.
\end{proof}



\begin{lemma} \label{lem:jensen}
\lean{Bounding.one_plus_cos_mul_one_plus_cos_ge}
\leanok
    For all $a,b \in \R$ with $|a|,|b|\leq 2$ the following inequality holds:
    \[
        (1+\cos(a))(1+\cos(b))\geq 2+2\cos\Big(\sqrt{a^2+b^2}\Big),
    \]
    with equality only for $a=0$ or $b=0$.
\end{lemma}

\begin{proof}
Use the Jensen inequality. See \cite{polyhedron.without.rupert}, Lemma 11.
\end{proof}

\begin{lemma} \label{lem:RxRy}
\lean{Bounding.norm_RxRy_minus_id_le}
\leanok
For any $\alpha,\beta\in \mathbb{R}$ one has
\[
    \|R_x(\alpha)R_y(\beta)-\id\| \leq  \sqrt{\alpha^2+\beta^2}
\]
with equality only for $\alpha = \beta = 0$.
\end{lemma}

\begin{proof}
\uses{lem:jensen, lem:RaRa}
See \cite{polyhedron.without.rupert}, Lemma 12.
\end{proof}

\begin{lemma} \label{lem:sqrt2}
\lean{Bounding.norm_M_sub_lt}
\leanok
Let $\epsilon>0$ and $|\theta-\thetab|,|\varphi-\phib| \leq \varepsilon$ then
$\|M(\theta, \phi)-M(\thetab,\phib)\|, \|X({\theta, \varphi})-X(\thetab,\phib)\| < \sqrt{2}\varepsilon.$
\end{lemma}

\begin{proof}
\uses{lem:RxRy}
See \cite{polyhedron.without.rupert}, Lemma 13.
\end{proof}

\begin{lemma} \label{lem:XPgt0}
  \lean{Bounding.XPgt0}
  \leanok
    Let $P \in \R^3$ with $\|P\| \leq 1$. Further, let $\epsilon>0$ and $\thetab,\phib, \theta, \phi \in \R$ such that $|\thetab-\theta|, |\phib - \phi| \leq \epsilon$. If
    \(
        \langle X(\thetab,\phib),P \rangle>\sqrt{2}\varepsilon
    \)
    then
    \(
        \langle X(\theta, \phi),P \rangle>0.
    \)
\end{lemma}

\begin{proof}
\uses{lem:sqrt2}
See \cite{polyhedron.without.rupert}, Lemma 14.
\end{proof}

\begin{lemma} \label{lem:MPgtr}
\lean{Bounding.norm_M_apply_gt}
\leanok
    Let $P \in \R^3$ with $\|P\| \leq 1$. Further, let $\epsilon, r>0$ and $\thetab,\phib, \theta, \phi \in \R$ such that $|\thetab-\theta|, |\phib - \phi| \leq \epsilon$. If
    \(
        \| M(\thetab,\phib) P \| > r + \sqrt{2}\varepsilon
    \)
    then
    \(
        \| M(\theta,\phi) P \| > r.
    \)
\end{lemma}

\begin{proof}
\uses{lem:sqrt2}
See \cite{polyhedron.without.rupert}, Lemma 15.
\end{proof}

\begin{lemma} \label{lem:sqrt5}
  \lean{Bounding.norm_RM_sub_RM_le}
  \leanok
    Let $\epsilon>0$ and $|\theta-\thetab|,|\varphi-\phib|,|\alpha-\alphab|\leq\varepsilon$ then
    $\|R(\alpha) M(\theta, \phi)-R(\alphab)M(\thetab,\phib)\| < \sqrt{5} \varepsilon.$
\end{lemma}

\begin{proof}
\uses{lem:sqrt2, lem:RxRy}
See \cite{polyhedron.without.rupert}, Lemma 16.
\end{proof}
