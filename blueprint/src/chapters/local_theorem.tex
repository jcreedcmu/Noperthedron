\chapter{The Local Theorem}

\begin{lemma} \label{lem:pythagoras}
  \lean{Local.pythagoras}
  \leanok
    For any $P \in \mathbb{R}^3$ one has
    $\big\|M(\theta, \phi) P\big\|^2=\|P\|^2-\langle X({\theta,\varphi}),P\rangle^2$.
\end{lemma}

\begin{proof}
\leanok
See \cite{polyhedron.without.rupert}, Lemma 21.
\end{proof}

\begin{definition} \label{def:spanp}
  \lean{Local.Spanp}
  \leanok
      Given $v_1, \dots, v_n \in \R^n$ write $\mathrm{span}^+(v_1,\dots,v_n)$ for the set (simplicial cone) in $\R^n$ defined by
    \[
        \mathrm{span}^+(v_1,\dots,v_n) = \Big\{ w \in \R^n \colon \exists \lambda_1,\dots,\lambda_n > 0 \text{ s.t. } w = \sum_{i=1}^n \lambda_i v_i \Big\},
    \]
    which is the natural restriction of $\mathrm{span}(v_1,\dots,v_n)$ to positive weights.
\end{definition}

\begin{lemma} \label{lem:langles}
  \lean{Local.langles}
  \leanok
Let $V_1,V_2,V_3,Y,Z \in \R^3$ with $\|Y \|=\| Z \|$ and $Y,Z \in \mathrm{span}^+(V_1,V_2,V_3)$. Then at least one of the following inequalities does not hold:
\begin{align*}
    \langle V_1, Y \rangle > \langle V_1, Z \rangle,\\
    \langle V_2, Y \rangle > \langle V_2, Z \rangle,\\
    \langle V_3, Y \rangle > \langle V_3, Z \rangle.
\end{align*}
\end{lemma}

\begin{proof}
\leanok
See \cite{polyhedron.without.rupert}, Lemma 23.
\end{proof}

\begin{lemma} \label{lem:scalarprodbars}
\lean{Local.abs_sub_inner_bars_le}
\leanok
    For $A,\overline{A},B,\overline{B}\in \R^{n\times n}$ and $P_1,P_2\in \R^n$ it holds that
    \[
        |\langle AP_1,BP_2\rangle-\langle \overline{A}P_1,\overline{B}P_2\rangle|\leq \|P_1\|\cdot \|P_2\|\cdot \Big( \|A-\overline{A}\|\cdot \|\overline{B}\| +  \|\overline{A}\|\cdot \|B-\overline{B}\|+\|A-\overline{A}\|\cdot \|B-\overline{B}\|\Big).
    \]
\end{lemma}

\begin{proof}
\leanok
See \cite{polyhedron.without.rupert}, Lemma 24.
\end{proof}

\begin{lemma} \label{lem:absscalar}
\lean{Local.abs_sub_inner_le}
\leanok
    For $A,B\in \R^{n\times n}$ and $P_1,P_2\in \R^n$ one has
    $$|\langle AP_1,AP_2\rangle-\langle BP_1,BP_2\rangle|\leq \|P_1\|\cdot \|P_2\|\cdot \|A-B\|\cdot \bigg(\|A\|+\|B\| + \|A-B\|\bigg).$$
\end{lemma}

\begin{proof}
\leanok
See \cite{polyhedron.without.rupert}, Lemma 25.
\end{proof}

\begin{lemma} \label{lem:origintriangle}
\lean{Local.origin_in_triangle}
\leanok
    Let $A,B,C\in \mathbb{R}^2$ be such that $
    \langle R({\pi/2}) A,B\rangle,
    \langle R({\pi/2}) B,C\rangle,
    \langle R({\pi/2}) C,A\rangle >0$. Then the origin lies strictly in the triangle $ABC$.
\end{lemma}

\begin{proof}
\leanok
See \cite{polyhedron.without.rupert}, Lemma 26.
\end{proof}

\begin{definition}
  \label{def:eps-spanning}
  \lean{Local.Triangle.Spanning}
  \leanok
  Let $\theta, \varphi \in \mathbb{R}$, $\varepsilon > 0$, and set $M := M(\theta, \varphi)$.
  Three points $P_1, P_2, P_3 \in \mathbb{R}^3$ with $\|P_1\|, \|P_2\|, \|P_3\| \leq 1$ are
  called $\varepsilon$-spanning for $(\theta, \varphi)$ if it holds that:
\begin{align*}
    \langle R(\pi/2) M P_1,M P_{2}\rangle > 2 \epsilon(\sqrt{2} + \varepsilon),\\
    \langle R(\pi/2) M P_2,M P_{3}\rangle > 2 \epsilon(\sqrt{2} + \varepsilon),\\
    \langle R(\pi/2) M P_3,M P_{1}\rangle > 2 \epsilon(\sqrt{2} + \varepsilon).
\end{align*}
\end{definition}


\begin{lemma} \label{lem:eps-spanning}
  \uses{def:eps-spanning, def:spanp}
  \lean{Local.vecX_spanning}
  \leanok
    Let $P_1, P_2, P_3 \in \R^3$ with $\|P_1\|,\|P_2\|,\|P_3\| \leq 1$ be $\epsilon$-spanning for $(\thetab, \phib)$ and let $\theta, \phi \in \R$ such that $|\theta - \overline{\theta}|, |\phi - \overline{\phi}| \leq \epsilon$. Assume that $\langle X(\theta, \phi), P_i \rangle > 0$ for $i=1,2,3$. Then
    \[
        X(\theta, \phi) \in \spanp(P_1, P_2, P_3).
    \]
\end{lemma}

\begin{proof}
\uses{lem:scalarprodbars, lem:origintriangle, lem:sqrt2}
See \cite{polyhedron.without.rupert}, Lemma 28.
\end{proof}

\begin{lemma} \label{lem:inCirc}
  \lean{Local.inCirc}
  \leanok
    Let $P, Q \in \R^3$ with $\|P\|, \|Q\| \leq 1$. Let $\epsilon>0$ and $\thetab_1,\phib_1,\thetab_2,\phib_2,\alphab \in \R$, then set
    \[
        T \coloneqq \left(R(\alphab) M(\thetab_1, \phib_1) P + M(\thetab_2, \phib_2) Q\right)/2 \in \R^2,
    \]
    and $\delta \geq \|T - M(\thetab_2, \phib_2) Q\|$. Finally, let $\theta_1, \phi_1, \theta_2, \phi_2, \alpha \in \R$ with $|\thetab_1-\theta_1|, |\phib_1 - \phi_1|, |\thetab_2-\theta_2|, |\phib_2-\phi_2|, |\alphab - \alpha| \leq \epsilon$. Then $R(\alpha)M(\theta_1, \phi_1) P, M(\theta_2, \phi_2) Q \in \Circ_{\delta + \sqrt{5} \epsilon}(T)$.
\end{lemma}

\begin{proof}
\uses{lem:sqrt2, lem:sqrt5}
\leanok
See \cite{polyhedron.without.rupert}, Lemma 30.
\end{proof}

\begin{definition}
  \label{def:LMD}
  \lean{Local.LocallyMaximallyDistant}
  \leanok
    Let $\PP \subset \R^2$ be a convex polygon and $Q \in \PP$ one of its vertices. Assume that for some $\overline{Q} \in \R^2$ it holds that $Q \in \Circ_{\delta}(\overline{Q})$, i.e. $\|Q - \overline{Q}\| < \delta$. Define $\Sect_\delta(\overline{Q}) \coloneqq \Circ_{\delta}(\overline{Q}) \cap \PP^\circ$ as the intersection between $\Circ_{\delta}(\overline{Q})$ and the interior of the convex hull of $\PP$.

    Moreover, $Q \in \PP$ is called \emph{$\delta$-locally maximally distant with respect to $\overline{Q}$ ($\delta$-LMD$(\overline{Q})$)} if for all $A \in \Sect_\delta(\overline{Q})$ it holds that $\|Q\| > \|A\|$.
\end{definition}

\begin{lemma} \label{lem:LMD}
  \uses{def:LMD}
  \lean{Local.inner_ge_implies_LMD}
  \leanok
    Let $\PP$ be a convex polygon and $Q \in \PP$ be one of its vertices. Let $\overline{Q} \in \R^2$ with $\|Q - \overline{Q}\| < \delta$ for some $\delta>0$. Assume that for some $r > 0 $ such that $\|Q\| > r$ it holds that
    \[
        \frac{\langle Q, Q - P_j \rangle}{\|Q\|\|Q - P_j\|} \geq \delta/r,
    \]
    for all other vertices $P_j \in \PP \setminus Q$. Then $Q \in \PP$ is $\delta$-locally maximally distant with respect to $\overline{Q}$.
\end{lemma}

\begin{proof}
See \cite{polyhedron.without.rupert}, Lemma 32.
\end{proof}

\begin{lemma} \label{lem:coss}
  \lean{Local.coss}
  \leanok
  Let $\epsilon>0$ and $\theta,\thetab, \phi, \phib \in \R$ with $|\theta - \overline{\theta}|, |\phi - \overline{\phi}| \leq \epsilon$. Define $M = M(\theta, \phi)$ and $\overline{M} = M(\thetab, \phib)$ and let $P, Q \in \R^3$ with $\|P\|, \|Q\| \leq 1$. Assume that
    \[
        \frac{\langle \overline{M} P,\overline{M} (P-Q)\rangle - 2 \epsilon \|P-Q\| \cdot  (\sqrt{2}+\varepsilon)}{ \big(\|\overline{M} P\|+\sqrt{2} \varepsilon \big) \cdot \big(\|\overline{M}(P-Q)\|+2 \sqrt{2} \varepsilon\big)} > 0.
    \]
  Then:
    \[
        \frac{\langle {M} P,{M} (P-Q)\rangle}{\|{M} P\| \cdot \|{M}(P-Q)\|} \geq
        \frac{\langle \overline{M} P,\overline{M} (P-Q)\rangle - 2 \epsilon \|P-Q\| \cdot  (\sqrt{2}+\varepsilon)}{ \big(\|\overline{M} P\|+\sqrt{2} \varepsilon \big) \cdot \big(\|\overline{M}(P-Q)\|+2 \sqrt{2} \varepsilon\big)}.
    \]
\end{lemma}
\begin{proof}
\uses{lem:absscalar, lem:sqrt2}
\leanok
See \cite{polyhedron.without.rupert}, Lemma 33.
\end{proof}

\begin{theorem}[Local Theorem] \label{thm:local}
\leanok
\lean{Local.local_theorem}
    Let $\PPP$ be a polyhedron with radius $\rho=1$ and $P_1, P_2, P_3, Q_1, Q_2, Q_3 \in \PPP$ be not necessarily distinct. Assume that $P_1, P_2, P_3$ and $Q_1, Q_2, Q_3$ are congruent.

    Let $\epsilon>0$ and $\thetab_1,\phib_1,\thetab_2,\phib_2,\alphab \in \R$, then set $\Xib \coloneqq X(\thetab_1,\phib_1), \Xiib \coloneqq X(\thetab_2,\phib_2)$ as well as $\Mib \coloneqq M(\thetab_1,\phib_1), \Miib \coloneqq M(\thetab_2,\phib_2)$.
    Assume that there exist $\sigma_P, \sigma_Q \in \{0,1\}$ such that
    \[
        (-1)^{\sigma_P} \langle \Xib,P_i\rangle>\sqrt{2}\varepsilon \quad \text{and} \quad
        (-1)^{\sigma_Q} \langle \Xiib , Q_i\rangle>\sqrt{2}\varepsilon, \tag{A$_\epsilon$}
    \]
    for all $i=1,2,3$.
    Moreover, assume that $P_1,P_2,P_3$ are $\epsilon$-spanning for $(\thetab_1,\phib_1)$ and that $Q_1,Q_2,Q_3$ are $\epsilon$-spanning for $(\thetab_2,\phib_2)$.
    Finally, assume that for all $i = 1,2,3$ and any $Q_j \in \PPP \setminus Q_i$ it holds~that
    \[
        \frac{\langle \Miib Q_i,\Miib (Q_i-Q_j)\rangle - 2 \epsilon \|Q_i-Q_j\| \cdot  (\sqrt{2}+\varepsilon)}{ \big(\|\Miib Q_i\|+\sqrt{2} \varepsilon \big) \cdot \big(\|\Miib(Q_i-Q_j)\|+2 \sqrt{2} \varepsilon\big)} > \frac{\sqrt{5} \epsilon + \delta}{r}, \tag{B$_\epsilon$}
    \]
    for some $r >0$ such that $\min_{i=1,2,3}\| \Miib Q_i \| > r + \sqrt{2} \epsilon$ and for some $\delta \in \R$ with
    \[
        \delta \geq \max_{i=1,2,3}\left\|R(\alphab) \Mib P_i - \Miib Q_i\right\|/2.
    \]
    Then there exists no solution to Rupert's problem $R(\alpha) M(\theta_1,\phi_1)\PPP \subset  M(\theta_2,\phi_2)\PPP^\circ$ with
    \[
        (\theta_1, \phi_1, \theta_2, \phi_2, \alpha) \in [\thetab_1\pm\epsilon,\phib_1\pm\epsilon,\thetab_2\pm\epsilon,\phib_2\pm\epsilon,\alphab\pm\epsilon] \coloneqq U \subseteq \R^5.
    \]
\end{theorem}

\begin{proof}
\uses{lem:langles,lem:XPgt0,lem:eps-spanning,lem:MPgtr,lem:inCirc,lem:coss,lem:LMD,lem:pythagoras}
See \cite{polyhedron.without.rupert}, Theorem 36.
\end{proof}
